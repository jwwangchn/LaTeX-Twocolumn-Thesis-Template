\section{基于像素的最优分割算法}

本节主要介绍一种基于像素的面向目标提取最优分割算法SRM(Statistical Region Merging),即基于统计学的区域合并算法。

\subsection{算法简介}

基于统计学的区域合并算法SRM是一种像素级的区域合并算法,它的核心思想是将具有相同统计特性的像素进行合并,进而得到面向目标提取的图像分割结果。

\subsection{算法原理}

SRM算法的核心可以概括为两条准则:1. 在 RGB 颜色空间的任何一个颜色通道中,同一统计学区域中的像素具有相同的统计期望值。2. 相邻统计学区域的统计期望值在 RGB 颜色空间中至少一个通道不同。

SRM算法的执行步骤如下:

1. 在图像$I$中,定义集合$S_I$为$I$的四邻域像素集合。在本章节,定义符号$|\cdot|$表示像素数量。

2. 假设像素$p$和像素$p'$为相邻的两个像素,在判断两个像素是否属于同一区域的,需要计算两像素统计学相似度,通过函数$f$来计算,$f$的定义如下:
$$
P_i=f_a(p,p')=|\overline{N_p(p')_a}-\overline{N_{p'}(p)_a}|
$$
其中,$N_p(p')$为与像素$p'$的距离小于常数$2\Delta$的区域。通过$f$计算好相似度之后,将相似度按升序排列。

3. 通过$P(R,R')$判断区域$R$和区域$R'$是否应该合并,$P(R,R')$定义如下:

\begin{displaymath}
P(R,R') = \left\{ \begin{array}{ll}

true & \textrm{$|\overline{R_a}-\overline{R_a'}|\leq \sqrt{b^2{R}+b^2{R'}}$}\\
false & \textrm{otherwise}

\end{array} \right.
\end{displaymath}


其中,$a$表示RGB颜色空间中${R,G,B}$三个颜色通道。$b(R)$定义如下:
$$
b(R,R' )=g\sqrt{\frac{1}{2Q|R|}\ln{(R_{|R|}/\delta)}}
$$
$Q$为定义的常量,$\delta=1/6|I|^2$,$g=256$。

对经过排序的相邻像素对$(p,p')$如果$R(p)=R(p')$即像素$p$和像素$p'$不在同一个区域,则计算$P(R(p),R(p'))$,返回值为$true$时,合并$R(p)$和$R(p')$。

迭代上述过程,当图像中所有像素都经过上述运算之后,得到最终的合并结果。
