%%%%%%%%%%%%%%%%%%%%%%%%%%%%%%%%%%%%%%%%%%%%%%%%%%%%%%%%%%%%%%%%

%%%%%%%%%%%%%%%%%%%%%%%%%%%%%%%%%%%%%%%%%%%%%%%%%%%%%%%%%%%%%%%%
%  调整section名称与正文之间的距离
%%%%%%%%%%%%%%%%%%%%%%%%%%%%%%%%%%%%%%%%%%%%%%%%%%%%%%%%%%%%%%%%
无人机作为一种新的遥感平台和影像获取方法,具有高分辨率、高灵活性、高效率和低成本的优势,其应用范围越来越广。无人机航拍图像的分辨率可达厘米级,图像中的相邻像素往往具有很强的相关性,导致“同物异谱”,“异物同谱”现象大量发生,同时图像数据量大幅度增加,地物信息呈现高度细节化。因此图像解译的方式必须要从传统的面向逐个像元光谱信号处理的方式转变到以面向对象-超像素的空间、结构和几何信息处理为主的方式上来。本文主要围绕无人机航拍图像的超像素表达展开研究,并在此基础上探索面向目标提取的最优图像分割方法。

超像素是指具有相似感官特征的相邻像素所组成的像素块。与基于单个像素的分割算法相比,基于超像素的分割算法具有以下优势:1. 超像素提取了图像的局部特征,保留了对图像进行后续处理所需的有效信息;2. 超像素的使用能大大减小图像中的冗余信息,降低了图像处理算法的运行时间。

经过十余年的发展,超像素技术已经被广泛地应用于各大计算机视觉和图像处理领域。例如,目标跟踪、人体姿态估计、显著性分析等。
